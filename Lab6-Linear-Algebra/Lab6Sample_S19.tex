%% IMPORTANT: Once working, run latex 3 times to get listoffigures to work

%% Be sure to check spelling!

%% Put your name and the proper due date in place

%% Copy the lstinputlisting and figure code as many times as you need
%% Be sure to put in your own file names if appropriate

%% Note that the \epsfig and \lstinputlisting commands
%% are currently commented out with %%% - until the
%% files exist, processing this code without them will result in an error
%% so leave the comments until you have created the files!

\documentclass{article}
\usepackage{amsmath}    % loads AMS-Math package
\usepackage{epsfig}     % allows PostScript files
\usepackage{listings}   % allows lstlisting environment
\usepackage{moreverb}   % allows listinginput environment
\usepackage[letterpaper, margin=0.75in]{geometry}  % set paper size/margins
\usepackage{EGR103S19}  % colorful file imports

\begin{document}
\begin{center}
\rule{6.5in}{0.5mm}\\~\\
\textbf{\large EGR 103L -- Spring 2019}\\~\\
\textbf{\huge Laboratory 6 - Linear Algebra}\\~\\
***NAME (NetID)***\\
***Lab Section N, DAY TIMES***\\
***DATE DUE***\\~\\
{\small I understand and have adhered to all the tenets of the Duke
  Community Standard in completing every part of this assignment.  I
  understand that a violation of any part of the Standard on any part
  of this assignment can result in failure of this assignment, failure
  of this course, and/or suspension from Duke University.} 
\rule{6.5in}{0.5mm}\\
\end{center}
\tableofcontents
\listoffigures
\pagebreak
\section{Based on Chapra Problem 8.3}
% Equation:
\begin{align*}
\begin{bmatrix}
 ? & ? & ?\\
 ? & ? & ?\\
 ? & ? & ?
\end{bmatrix}
\begin{bmatrix}
~x_1~ \\ x_2 \\ x_3
\end{bmatrix}&=
\begin{bmatrix}
? \\ ? \\ ? 
\end{bmatrix}
\end{align*}

The calculated solutions for $x_i$ are:
\begin{align*}
x&=\begin{bmatrix}
  ?\\
  ?\\
  ?
\end{bmatrix}
\end{align*}
The transpose and inverse of $A$ are:
\begin{align*}
A' &=
\begin{bmatrix}
 ? & ? & ? \\
 ? & ? & ? \\
 ? & ? & ? \\
\end{bmatrix} &
\mbox{inv}(A) &= 
\begin{bmatrix}
 ? & ? & ? \\
 ? & ? & ? \\
 ? & ? & ? \\
\end{bmatrix}
\end{align*}
The condition numbers of $A$ are:
%% present all four different condition numbers in a profesional looking way

%% discuss meaning of condition numbers

\section{Chapra Problem 8.10}
The matrix equation can be written as:
\begin{align*}
\begin{bmatrix}
\cos(30^o) & 0 & -\cos(60^o) & 0 & 0 & 0\\
? & ? & ? & ? & ? & ?\\
? & ? & ? & ? & ? & ?\\
? & ? & ? & ? & ? & ?\\
? & ? & ? & ? & ? & ?\\
? & ? & ? & ? & ? & ?\\
\end{bmatrix}
\begin{bmatrix}
~F_1~ \\ F_2 \\ F_3 \\ H_2 \\ V_2 \\ V_3
\end{bmatrix}&=
\begin{bmatrix}
F_{1,h} \\ ? \\ ? \\ ? \\ ? \\ ? % can use F_{1,\mbox{h}} too
\end{bmatrix}
\end{align*}

The solutions written into the text file are:
%\lstinputlisting{truss_data.txt}
% ^ take out comment when file exists
%% Did you use -2000 for F1v?  Check.  Make sure you used -2000!

\section{Chapra Problem 8.16}
The scripts and plot are in the appendices.

\section{Parameter Sweep 1}
%% Discuss relationship between i and v

\section{Parameter Sweep 2}
%% Discuss relationship between i and R
%% Discuss relationahip between log10(cond) and R

\pagebreak
\appendix
\section{Codes and Output}
% Put the name of your file in the subsection name 
% and the listinginput input
% Be sure to include the community standard in codes!
% Add \pagebreaks if they make sense
% Remember to take out comments to include files!

\lstset{style=python103, language=python}

\subsection{Chapra 8.3}
%\lstinputlisting{chapra_08_03.py}
\subsection{Chapra 8.10}
%\lstinputlisting{chapra_08_10.py}
\subsection{Chapra 8.16}
%\lstinputlisting{chapra_08_16.py}
\subsection{Creative}
%\lstinputlisting{creative_chapra_08_16.py}
\subsection{Sweep 1}
%\lstinputlisting{sweep_1.py}
\subsection{Sweep 2}
%\lstinputlisting{sweep_2.py}

\pagebreak
\section{Figures}

\begin{figure}[htb!]
\begin{center}
%\epsfig{file=NAME.eps, scale=0.7}
\caption{Creative Rotated Shape}
\end{center}
\end{figure}

\begin{figure}[htb!]
\begin{center}
%\epsfig{file=NAME.eps, scale=0.7}
\caption{Current vs. Voltage}
\end{center}
\end{figure}

\begin{figure}[htb!]
\begin{center}
%\epsfig{file=NAME.eps, scale=0.7}
\caption{Current vs. Resistance}
\end{center}
\end{figure}

\begin{figure}[htb!]
\begin{center}
%\epsfig{file=NAME.eps, scale=0.7}
\caption{Condition Number vs. Resistance}
\end{center}
\end{figure}

\end{document}
